\documentclass{article}

\usepackage[latin1]{inputenc}
\usepackage[italian]{babel}
\usepackage[pdftex]{graphicx}
\graphicspath{{../img/}}

%la novit� � questo pacchetto che serve a creare subfigure
\usepackage{subfig}
\usepackage{epstopdf}

\begin{document}

\section{Immagini allineate}

\subsection{L'ambiente \texttt{minipage}}

	\begin{figure}
    	\begin{minipage}{0.4\textwidth}
        	\centering
        	\includegraphics[width=\textwidth]{AIM}
        	\caption{Questo � il logo .eps}
        	\label{fig:minipage1}
	   \end{minipage}
    	\hfill
    	\begin{minipage}{0.4\textwidth}
        	\centering
        	\includegraphics[width=\textwidth]{AIM_eps_falso}
        	\caption{Questo � il logo .jpg}
        	\label{fig:minipage2}
    	\end{minipage}
    	\caption{Formati grafici con \texttt{minipage}.}
    	\label{fig:minipage}
	\end{figure}
	
	Un esempio di minipage in figura \ref{fig:minipage}: il logo in formato .eps in figura \ref{fig:minipage1} e in formato .jpg in figura \ref{fig:minipage2}.

\subsection{L'ambiente \texttt{subfloat}}

	\begin{figure}
	\centering
	\subfloat[Questo � il logo .eps]{
    	\label{fig:subfloat1}
    	\includegraphics[width=0.4\textwidth]{AIM}
	}
	\qquad
	\subfloat[Questo � il logo .jpg]{
    	\label{fig:subfloat2}
    	\includegraphics[width=0.4\textwidth]{AIM_eps_falso}
	}
	\caption{Formati grafici con \texttt{subfloat}.}
	\label{fig:subfloat}
	\end{figure}

	Un esempio di subfloat in figura \ref{fig:subfloat}: il logo in formato .eps in figura \ref{fig:subfloat1} e in formato .jpg in figura \ref{fig:subfloat2}.
    
    Notate la differenza nei riferimenti.

\end{document} 