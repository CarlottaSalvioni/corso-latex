\documentclass[xcolor={dvipsnames}]{beamer}
%xcolor={dvipsnames} è per utilizzare colori non standard.
%Tabella completa dei colori: http://en.wikibooks.org/wiki/LaTeX/Colors

% Stile per Beamer
\usetheme{Warsaw}
\usecolortheme[named=NavyBlue]{structure}

\setbeamertemplate{caption}[numbered] % per stampare il numero di etichetta delle immagini
\setbeamertemplate{theorems}[numbered] % per stampare il numero di etichetta dei teoremi

% Pacchetti
\usepackage[italian]{babel}
\usepackage[utf8]{inputenc}
\usepackage{amsmath}
\usepackage{amsthm}

%Per i links
\usepackage{hyperref}
\usepackage{graphicx}
\graphicspath{{img/}}

\beamertemplatearrowbibitems % frecce accanto alle voci della bibliografia

% Informazioni sulla presentazione
\title{La nostra prima presentazione in \LaTeX \\ con titolo su due righe}
\author[Autori sintetici]{Autori in versione estesa}
\date{Data della presentazione}
\institute[Polimi]{Politecnico di Milano}
\titlegraphic{\includegraphics[scale=0.05]{logo-aim}}

%Se volete togliere le reference!!!
%\usepackage[bf,footnotesize,center]{caption}
%\captionsetup{labelformat=empty,labelsep=none}

\begin{document}

\titlepage

%%%%%%%%%%%%%%%%%%%%%%%%%%%%%%%%%%%%%%%%%%%%%%%%%%%%%%%%%%%%%%%%%%%%%%%%%%%%

\section{Prima sezione}
\begin{frame}{Questa è la prima slide}
    È possibile inserire tutto ciò che abbiamo già visto (e anche di più!):

    \begin{itemize}[<+->] %Appare un elemento per volta.
     \item Elenchi puntati o numerati
     \item equazioni: $\frac{\partial \mathbf{U}}{\partial t} + \frac{\partial \mathbf{F}(\mathbf{U})}{\partial x} = \mathbf{f}$;
     \item[@] link: \url{http://www.aim-mate.it};
     \item teoremi, dimostrazioni;
     \item tabelle, immagini, \textbf{filmati};
     \item bibliografia;
     \item everything else \ldots
    \end{itemize}


\end{frame}

%%%%%%%%%%%%%%%%%%%%%%%%%%%%%%%%%%%%%%%%%%%%%%%%%%%%%%%%%%%%%%%%%%%%%%%%%%%%%

\section{Seconda sezione}
\subsection{Prima sottosezione}
\begin{frame}{Una slide con il sottotitolo}
\framesubtitle{Questo è il sottotitolo}

    Sono presenti tutti gli ambienti matematici che vengono racchiusi in blocchi predefiniti:

    \begin{theorem}
    \label{teorema:prova}
    Questo è l'enunciato di un teorema.
    \end{theorem}
    \begin{proof}
    Questa ne è la dimostrazione.
    \end{proof}
\end{frame}

%%%%%%%%%%%%%%%%%%%%%%%%%%%%%%%%%%%%%%%%%%%%%%%%%%%%%%%%%%%%%%%%%%%%%%%%%%%%%%%

\subsection{Seconda sottosezione}
\begin{frame}{Una figura}

  \begin{figure}[h]
  \includegraphics[height=0.3\textheight]{AIM.eps}
  \caption{Didascalia}
  \label{fig:AIM} % da mettere SEMPRE dopo \caption{}
  \end{figure}
  
  Si possono citare figure (Fig. \ref{fig:AIM}), teoremi (Teorema \ref{teorema:prova}), equazioni e ovviamente anche elementi della bibliografia \cite{artelatex}.
\end{frame}

%%%%%%%%%%%%%%%%%%%%%%%%%%%%%%%%%%%%%%%%%%%%%%%%%%%%%%%%%%%%%%%%%%%%%%%%
\begin{frame}{Utilizzo (eccessivo) di overlay}
% \onslide<n-> indica di mostrare il contenuto dall'overlay n in poi. Il valore n deve tenere conto anche dei \pause inseriti. È consigliabile di usare sempre una sola delle due opzioni - per slide -, a seconda delle necessità!
\onslide<1->Questo è un esempio di impostazione di overlay - anche all'interno di altri ambienti! \\
\begin{center}
\begin{tabular}{cccc}
	\onslide<2->a & b & c & d \\
	\onslide<3->e & f & g & h \\
	\onslide<4->i & k & l & m \\
\end{tabular}
\[ \onslide<5-> sin^2(x) + \onslide<6-> cos^2(x) \onslide<7-> = 1 \]
\end{center}
\onslide<8->
\begin{block}{Sull'(ab)uso di overlay}
Inserire gli overlay in maniera \textbf{coerente} con il contenuto della slide!
\end{block}
\end{frame}

\begin{frame}{Altri tipi di blocchi}
    \begin{exampleblock}{Esempio}
    Blocco per gli esempi.
    \end{exampleblock}
    
    \begin{alertblock}{Attenzione!}
    Blocco per gli alert.
    \end{alertblock}
    
    \begin{block}{}
    Un blocco può anche non avere un titolo (le graffe sono obbligatorie, anche se dentro sono vuote).
    \end{block}

\end{frame}

%%%%%%%%%%%%%%%%%%%%%%%%%%%%%%%%%%%%%%%%%%%%%%%%%%%%%%%%%%%%%%%%%%%%%%%%%

\begin{frame}{Bibliografia}
    \begin{thebibliography}{artelatex}
%	\addcontentsline{toc}{section}{Bibliografia}
	\bibitem[Pantieri 08]{artelatex} L. Pantieri, \emph{L'arte di scrivere con \LaTeX}. \url{http://www.lorenzopantieri.net/LaTeX_files/ArteLaTeX.pdf}. (2008).
	\bibitem[Esempio]{esempio} Autore, \emph{Titolo}. Rivista. 2015.
    \end{thebibliography}
\end{frame}

\end{document}
