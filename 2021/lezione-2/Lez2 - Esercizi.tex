\documentclass[10pt,a4paper]{article}
\usepackage[utf8]{inputenc}
\usepackage[T1]{fontenc}
\usepackage{amsmath, amssymb, amsthm}
\usepackage{amsfonts}
\usepackage{amssymb}
\usepackage{graphicx, float, caption, subcaption}
\usepackage{multirow, booktabs}
\usepackage[left=2.00cm, right=2.00cm, top=2.00cm, bottom=2.00cm]{geometry}
\author{Andrea Di Primio}
\title{Lezione 2: Esercizi}
\newtheorem{prop}{Proposition}[section]
\begin{document}
	\maketitle
\section{Esercizio 1: il nostro primo teorema (e sua dimostrazione)}
\begin{prop}[Vitali]
	Se $A \subset \mathbb{R}^n$ è un insieme aperto, allora esiste $B \subset A$ non misurabile secondo Lebesgue.
\end{prop}
\begin{proof}[Dimostrazione (solo un commento)]
	Una costruzione dell'insieme $B$, detto insieme di Vitali, non è banale e richiede l'uso dell'assioma della scelta.
\end{proof}

\section{Esercizio 2: un album fotografico}
\begin{figure}[H]
	\begin{subfigure}[t]{0.3\textwidth}
		\includegraphics[width=\textwidth]{example-image-a}
		\caption{Helsinki, 1915.}
	\end{subfigure} \hfill
	\begin{subfigure}[t]{0.3\textwidth}
		\includegraphics[width=\textwidth]{example-image-b}
		\caption{Ivrea, 1901.}
	\end{subfigure} \hfill
	\begin{subfigure}[t]{0.3\textwidth}
		\includegraphics[width=\textwidth]{example-image-c}
		\caption{Dublino, 1899.}
	\end{subfigure}
	\begin{subfigure}[t]{0.3\textwidth}
		\includegraphics[width=\textwidth]{example-image-a}
		\caption{Dresda, 1914.}
	\end{subfigure} \hfill
	\begin{subfigure}[t]{0.3\textwidth}
		\includegraphics[width=\textwidth]{example-image-b}
		\caption{Empoli, 1901.}
	\end{subfigure} \hfill
	\begin{subfigure}[t]{0.3\textwidth}
		\includegraphics[width=\textwidth]{example-image-c}
		\caption{Napoli, 1916.}
	\end{subfigure}
	\caption{Belle città!}
\end{figure}

\section{Esercizio 3: una tabella un po' particolare}
\begin{table}[H] 
	\begin{center}
		\begin{tabular}{|c|cc|cc|c|}
			\textbf{Col. 1} & \textbf{Col. 2} & \textbf{Col. 3} & \textbf{Col. 4} & \textbf{Col. 5} & \textbf{Col. 6}\\
			$\frac{2}{3}, \frac{1}{2},...$ & $\beta$ & $\gamma$ & $\delta^\xi$ & $\sin\lambda$ & $\log \pi$\\
			\hline 
			\multirow{3}{*}{Tre righe!} & a & b & c & d & e \\
			& f & g & h & i & j \\
			& k & l & m & n & o \\  % <-- Content of first column omitted.
			\hline
			\multicolumn{2}{|c|}{E questa?} & \multicolumn{4}{r}{Lascio aperto qui $\to$} \\
			\hline
			\multicolumn{3}{l}{$\to$ e rientro qui!} & \multicolumn{3}{c}{\texttt{bottomrule da 1 mm} $\downarrow$}\\
			\bottomrule[0.1cm]
		\end{tabular}
		\caption{Una tabella un po' particolare.}
	\end{center}
\end{table}
\section{Esercizio 4: l'occhio vuole la sua parte!}
\begin{minipage}[t][.3\textheight]{0.49\textwidth}
	La prima lettera dell'alfabeto è illustrata nella Figura seguente. Appare, nella lingua italiana, con una frequenza del 11.74\% circa (fonte Wikipedia).
	\vfill
	\begin{figure}[H]
		\includegraphics[width=\textwidth]{example-image-a}
	\end{figure}
\end{minipage} \hfill
\begin{minipage}[t][.3\textheight]{0.49\textwidth}
	Altre lettere includono la B e la C, meno frequenti, illustrate di seguito.
	\begin{figure}[H]
		\begin{subfigure}[t]{0.4\textwidth}
			\includegraphics[width=\textwidth]{example-image-b}
		\end{subfigure} \hfill
		\begin{subfigure}[t]{0.4\textwidth}
			\includegraphics[width=\textwidth]{example-image-c}
		\end{subfigure}
	\end{figure} \vfill
	La tabella riassume le loro frequenze alla seconda cifra decimale:
	\begin{table}[H]
		\centering
		\begin{tabular}{c|c}
			A & 11.74\% \\
			B & 0.92\% \\
			C & 4.50\% 
		\end{tabular}
	\end{table}
\end{minipage} \\
\end{document}