\documentclass{article}

\usepackage[italian]{babel}
\usepackage[pdftex]{graphicx}

%ricordiamo la seguente serve per specificare dove sono le immagini
\graphicspath{{../img/}}
%E questa per utilizzare immagini eps (bisogna aggiungere -shell-escape nel comando pdflatex)
\usepackage{epstopdf}

\begin{document}

\section{Immagini}

Proviamo ora a inserire una immagine qui \includegraphics{AIM}. Anche per le immagini � possibile creare dei riferimenti (ad esempio vedi figura \ref{img:logo}).

%Sintassi tipica
\begin{figure}
	\centering
	\includegraphics{AIM}
	\caption{La nostra prima immagine...}
	\label{img:logo}
\end{figure}

\clearpage

\begin{figure}
	\centering
	\includegraphics[width = \textwidth]{AIM}
	\caption{La nostra seconda immagine...}
	\label{img:logo2}
\end{figure}

\end{document} 