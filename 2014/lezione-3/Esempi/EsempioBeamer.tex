\documentclass[leqno]{beamer}

% Stile per Beamer
\usetheme{Warsaw}
% Pacchetti
\usepackage[italian]{babel}
\usepackage{amsmath}
\usepackage{amsfonts}
\usepackage{amsthm}
\usepackage{amstext}
\usepackage{alltt}
\usepackage{verbatim}
\usepackage{array}
\usepackage{booktabs}
%Per i links
\usepackage{hyperref}
\usepackage{graphicx}
\graphicspath{{../img/}}

\beamertemplatearrowbibitems

% Informazioni sulla presentazione

\title[La nostra prima presentazione in \LaTeX\ ]{La nostra prima presentazione in \LaTeX\ con un titolo su due righe}
\author[Autori sintetici]{Autori in versione estesa}
\date{Data della presentazione}
\institute[Polimi]{Politecnico di Milano}
\titlegraphic{\includegraphics[scale=0.1]{logo-aim}}
%Se volete togliere le reference!!!

%\usepackage[bf,footnotesize,center]{caption}
%\captionsetup{labelformat=empty,labelsep=none}

\begin{document}

\begin{frame}
\titlepage
\end{frame}
%%%%%%%%%%%%%%%%%%%%%%%%%%%%%%%%%%%%%%%%%%%%%%%%%%%%%%%%%%%%%%%%%%%%%%%%%%%%

\section{Prima sezione}
\begin{frame}
    \frametitle{Questa \`e la prima slide}

    Si possono inserire tutte le cose che abbiamo visto nei documenti in \LaTeX:

    \begin{itemize}[<+->]
     \item Elenchi puntati o numerati
     \item Equazioni: $\frac{\partial \mathbf{U}}{\partial t} + \eta \frac{\partial \mathbf{F}(\mathbf{U})}{\partial x} = \mathbf{5}$
     \item[@] Link: \url{http://www.mate.polimi.it/aim}
     \item Teoremi, dimostrazioni
     \item Tabelle, immagini, (filmati), \ldots
     \item Bibliografia (come vedremo alla fine)
     \item Everything else \ldots
    \end{itemize}


\end{frame}

%%%%%%%%%%%%%%%%%%%%%%%%%%%%%%%%%%%%%%%%%%%%%%%%%%%%%%%%%%%%%%%%%%%%%%%%%%%%%

\section{Seconda sezione}

\begin{frame}
    \frametitle{Una slide con anche il sottotitolo}
      \framesubtitle{Questo \`e il sottotitolo}

    Sono presenti tutti gli ambienti matematici che vengono racchiusi in blocchi predefiniti:

    \begin{theorem}
    \label{teo:smart2}
    Questo \`e l'enunciato di un teorema.
    
    \end{theorem}
\begin{proof}
    Questa ne \`e la dimostrazione.
    \end{proof}
    Si possono citare figure (Fig. \ref{fig:AIM}), teoremi (Teorema \ref{teo:smart2}), equazioni \ldots
    e ovviamente anche elementi della bibliografia \cite{artelatex}.

\end{frame}

%%%%%%%%%%%%%%%%%%%%%%%%%%%%%%%%%%%%%%%%%%%%%%%%%%%%%%%%%%%%%%%%%%%%%%%%%%%%%%%

\begin{frame}[fragile]
  \frametitle{Inseriamo uno script di codice}

  \begin{figure}[htbp]
   \includegraphics[height=0.3\textheight]{AIM.eps}
   \scriptsize{\caption{AIM}}
   \label{fig:AIM}
  \end{figure}

  \pause

  Per inserire l'immagine precedente serve il codice:
  {\footnotesize
  \begin{block}{titolo del blocco}
  \begin{verbatim}
  \begin{figure}[htbp]
   \includegraphics[width=0.5\textwidth]{AIM.eps}
   \caption{AIM}
   \label{fig:AIM}
  \end{figure}
  \end{verbatim}
  \end{block}
  }
\end{frame}
%%%%%%%%%%%%%%%%%%%%%%%%%%%%%%%%%%%%%%%%%%%%%%%%%%%%%%%%%%%%%%%%%%%%%%%%
\begin{frame}
\frametitle{altri tipi di blocchi}
  Questa slide serve per far vedere il comando \alert{alert}.
  inoltre abbiamo
  \begin{exampleblock}{esempio}
   per gli esempi
  \end{exampleblock}
  \begin{alertblock}{ATTENZIONE!!}
   attenti \`e in \alert{rosso}
  \end{alertblock}
  \begin{alertblock}{}
   Si pu\`o anche fare senza titolo.
  \end{alertblock}

\end{frame}

%%%%%%%%%%%%%%%%%%%%%%%%%%%%%%%%%%%%%%%%%%%%%%%%%%%%%%%%%%%%%%%%%%%%%%%%%

\begin{frame}
      \frametitle{Bibliografia}
      \begin{thebibliography}{introlatex}
%	\addcontentsline{toc}{section}{Bibliografia}

	\bibitem[Gorni 09]{introlatex} G. Gorni, \emph{Introduzione al \LaTeX}.  \url{http://users.dimi.uniud.it/~gianluca.gorni/TeX/itTeXdoc/CorsoTeXStampa.pdf}. (2009).
	\bibitem[Pantieri 08]{artelatex} L. Pantieri, \emph{L'arte di scrivere con \LaTeX}. \url{http://www.lorenzopantieri.net/LaTeX_files/ArteLaTeX.pdf}. (2008).
	
\end{thebibliography}

\end{frame}

\end{document}
