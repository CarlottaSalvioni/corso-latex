\documentclass[a4paper,10pt]{article}

\usepackage[utf8]{inputenc}
\usepackage[italian]{babel}

%Questo file va compilato direttamente con pdflatex/latex.
%Di solito l'editor (TeXStudio o altro) provvede a lanciare anche bibtex. Per rilanciarlo manualmente (su TeXStudio) andare su Tools->Bibliography.

%In caso, si può compilare manualmente utilizzando
%(pdf)latex
%bibtex
%(pdf)latex
%(pdf)latex

\title{Prova con bib\TeX}


\begin{document}

\maketitle

\newpage

\section{Vediamo che si può fare}
A titolo di esempio citiamo due libri molto famosi: \cite{Fey05} e anche \cite{Rudin87}.

\vspace{5cm}

%verranno visualizzate solo le voci citate.
%per visualizzare tutto decommentare la riga seguente:
%\nocite{*}

%questo specifica lo stile
\bibliographystyle{plain}

%questo specifica il file .bib da usare
\bibliography{sources}

\end{document}
