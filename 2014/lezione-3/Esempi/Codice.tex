\documentclass{article}


%I due pacchetti seguenti servono per aggungere l'ambiente lstlisting
%e aggiungere pezzi di codice nel documento
\usepackage{listings}
\usepackage[dvipsnames]{xcolor}


%La seguente istruzione serve per settare i paramentri di lstlisting
\lstset{
    inputencoding=utf8/latin1,
	rulecolor=\color{black},
    keywordstyle=\color{RoyalBlue},%\bfseries,
    basicstyle=\small\ttfamily,
    identifierstyle=\color{BrickRed},
    commentstyle=\color{Green}\ttfamily,
    stringstyle=\rmfamily,
    numbers=left,
    numberstyle=\scriptsize,%\tiny
    stepnumber=5,
    numbersep=8pt,
    showstringspaces=true,
    breaklines=true,
    frameround=ftff,
    frame=single
} 

\begin{document}

	%Qua viene usato lstlisting
	\begin{lstlisting}[language=matlab]
disp('Loop di Newton per l''equazione di Poisson')
errN  = 1;
iterN = 0;
while errN > tollN % *(1 - exp(-0.05*iterN))
    iterN = iterN + 1;
    rho = q*(nintr*(exp((phip(:,k)-psi(:,k))/Vth) - exp((psi(:,k)-phin(:,k))/Vth)) + doping);
    rho(setdiff(nOX,nI)) = 0;
    F = A_epstot*psi(:,k) - Identity*rho;
    coeff = q*nintr/Vth*(exp((psi(:,k)-phin(:,k))/Vth) + exp((phip(:,k)-psi(:,k))/Vth));
    coeff(setdiff(nOX,nI)) = 0;
    A = A_epstot + spdiags(Identity*coeff,0,nv,nv);
    delta_psi = -A(nodes_inter,nodes_inter)\F(nodes_inter);
    tk = 1; % - exp(-0.05*iterN); % damping
    psi(nodes_inter,k) = psi(nodes_inter,k) + tk*delta_psi;
    errN = norm(delta_psi,inf);
end
	\end{lstlisting}

% Se il codice sorgente � disponibile in un file, si pu� includere direttamente con il comando:
% \lstinputlisting[language=matlab]{main.m}
	
\end{document}
